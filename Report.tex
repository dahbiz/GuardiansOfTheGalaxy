\documentclass[11pt]{article}

    \usepackage[breakable]{tcolorbox}
    \usepackage{parskip} % Stop auto-indenting (to mimic markdown behaviour)
    

    % Basic figure setup, for now with no caption control since it's done
    % automatically by Pandoc (which extracts ![](path) syntax from Markdown).
    \usepackage{graphicx}
    % Maintain compatibility with old templates. Remove in nbconvert 6.0
    \let\Oldincludegraphics\includegraphics
    % Ensure that by default, figures have no caption (until we provide a
    % proper Figure object with a Caption API and a way to capture that
    % in the conversion process - todo).
    \usepackage{caption}
    \DeclareCaptionFormat{nocaption}{}
    \captionsetup{format=nocaption,aboveskip=0pt,belowskip=0pt}

    \usepackage{float}
    \floatplacement{figure}{H} % forces figures to be placed at the correct location
    \usepackage{xcolor} % Allow colors to be defined
    \usepackage{enumerate} % Needed for markdown enumerations to work
    \usepackage{geometry} % Used to adjust the document margins
    \usepackage{amsmath} % Equations
    \usepackage{amssymb} % Equations
    \usepackage{textcomp} % defines textquotesingle
    % Hack from http://tex.stackexchange.com/a/47451/13684:
    \AtBeginDocument{%
        \def\PYZsq{\textquotesingle}% Upright quotes in Pygmentized code
    }
    \usepackage{upquote} % Upright quotes for verbatim code
    \usepackage{eurosym} % defines \euro

    \usepackage{iftex}
    \ifPDFTeX
        \usepackage[T1]{fontenc}
        \IfFileExists{alphabeta.sty}{
              \usepackage{alphabeta}
          }{
              \usepackage[mathletters]{ucs}
              \usepackage[utf8x]{inputenc}
          }
    \else
        \usepackage{fontspec}
        \usepackage{unicode-math}
    \fi

    \usepackage{fancyvrb} % verbatim replacement that allows latex
    \usepackage{grffile} % extends the file name processing of package graphics
                         % to support a larger range
    \makeatletter % fix for old versions of grffile with XeLaTeX
    \@ifpackagelater{grffile}{2019/11/01}
    {
      % Do nothing on new versions
    }
    {
      \def\Gread@@xetex#1{%
        \IfFileExists{"\Gin@base".bb}%
        {\Gread@eps{\Gin@base.bb}}%
        {\Gread@@xetex@aux#1}%
      }
    }
    \makeatother
    \usepackage[Export]{adjustbox} % Used to constrain images to a maximum size
    \adjustboxset{max size={0.9\linewidth}{0.9\paperheight}}

    % The hyperref package gives us a pdf with properly built
    % internal navigation ('pdf bookmarks' for the table of contents,
    % internal cross-reference links, web links for URLs, etc.)
    \usepackage{hyperref}
    % The default LaTeX title has an obnoxious amount of whitespace. By default,
    % titling removes some of it. It also provides customization options.
    \usepackage{titling}
    \usepackage{longtable} % longtable support required by pandoc >1.10
    \usepackage{booktabs}  % table support for pandoc > 1.12.2
    \usepackage{array}     % table support for pandoc >= 2.11.3
    \usepackage{calc}      % table minipage width calculation for pandoc >= 2.11.1
    \usepackage[inline]{enumitem} % IRkernel/repr support (it uses the enumerate* environment)
    \usepackage[normalem]{ulem} % ulem is needed to support strikethroughs (\sout)
                                % normalem makes italics be italics, not underlines
    \usepackage{mathrsfs}
    

    
    % Colors for the hyperref package
    \definecolor{urlcolor}{rgb}{0,.145,.698}
    \definecolor{linkcolor}{rgb}{.71,0.21,0.01}
    \definecolor{citecolor}{rgb}{.12,.54,.11}

    % ANSI colors
    \definecolor{ansi-black}{HTML}{3E424D}
    \definecolor{ansi-black-intense}{HTML}{282C36}
    \definecolor{ansi-red}{HTML}{E75C58}
    \definecolor{ansi-red-intense}{HTML}{B22B31}
    \definecolor{ansi-green}{HTML}{00A250}
    \definecolor{ansi-green-intense}{HTML}{007427}
    \definecolor{ansi-yellow}{HTML}{DDB62B}
    \definecolor{ansi-yellow-intense}{HTML}{B27D12}
    \definecolor{ansi-blue}{HTML}{208FFB}
    \definecolor{ansi-blue-intense}{HTML}{0065CA}
    \definecolor{ansi-magenta}{HTML}{D160C4}
    \definecolor{ansi-magenta-intense}{HTML}{A03196}
    \definecolor{ansi-cyan}{HTML}{60C6C8}
    \definecolor{ansi-cyan-intense}{HTML}{258F8F}
    \definecolor{ansi-white}{HTML}{C5C1B4}
    \definecolor{ansi-white-intense}{HTML}{A1A6B2}
    \definecolor{ansi-default-inverse-fg}{HTML}{FFFFFF}
    \definecolor{ansi-default-inverse-bg}{HTML}{000000}

    % common color for the border for error outputs.
    \definecolor{outerrorbackground}{HTML}{FFDFDF}

    % commands and environments needed by pandoc snippets
    % extracted from the output of `pandoc -s`
    \providecommand{\tightlist}{%
      \setlength{\itemsep}{0pt}\setlength{\parskip}{0pt}}
    \DefineVerbatimEnvironment{Highlighting}{Verbatim}{commandchars=\\\{\}}
    % Add ',fontsize=\small' for more characters per line
    \newenvironment{Shaded}{}{}
    \newcommand{\KeywordTok}[1]{\textcolor[rgb]{0.00,0.44,0.13}{\textbf{{#1}}}}
    \newcommand{\DataTypeTok}[1]{\textcolor[rgb]{0.56,0.13,0.00}{{#1}}}
    \newcommand{\DecValTok}[1]{\textcolor[rgb]{0.25,0.63,0.44}{{#1}}}
    \newcommand{\BaseNTok}[1]{\textcolor[rgb]{0.25,0.63,0.44}{{#1}}}
    \newcommand{\FloatTok}[1]{\textcolor[rgb]{0.25,0.63,0.44}{{#1}}}
    \newcommand{\CharTok}[1]{\textcolor[rgb]{0.25,0.44,0.63}{{#1}}}
    \newcommand{\StringTok}[1]{\textcolor[rgb]{0.25,0.44,0.63}{{#1}}}
    \newcommand{\CommentTok}[1]{\textcolor[rgb]{0.38,0.63,0.69}{\textit{{#1}}}}
    \newcommand{\OtherTok}[1]{\textcolor[rgb]{0.00,0.44,0.13}{{#1}}}
    \newcommand{\AlertTok}[1]{\textcolor[rgb]{1.00,0.00,0.00}{\textbf{{#1}}}}
    \newcommand{\FunctionTok}[1]{\textcolor[rgb]{0.02,0.16,0.49}{{#1}}}
    \newcommand{\RegionMarkerTok}[1]{{#1}}
    \newcommand{\ErrorTok}[1]{\textcolor[rgb]{1.00,0.00,0.00}{\textbf{{#1}}}}
    \newcommand{\NormalTok}[1]{{#1}}

    % Additional commands for more recent versions of Pandoc
    \newcommand{\ConstantTok}[1]{\textcolor[rgb]{0.53,0.00,0.00}{{#1}}}
    \newcommand{\SpecialCharTok}[1]{\textcolor[rgb]{0.25,0.44,0.63}{{#1}}}
    \newcommand{\VerbatimStringTok}[1]{\textcolor[rgb]{0.25,0.44,0.63}{{#1}}}
    \newcommand{\SpecialStringTok}[1]{\textcolor[rgb]{0.73,0.40,0.53}{{#1}}}
    \newcommand{\ImportTok}[1]{{#1}}
    \newcommand{\DocumentationTok}[1]{\textcolor[rgb]{0.73,0.13,0.13}{\textit{{#1}}}}
    \newcommand{\AnnotationTok}[1]{\textcolor[rgb]{0.38,0.63,0.69}{\textbf{\textit{{#1}}}}}
    \newcommand{\CommentVarTok}[1]{\textcolor[rgb]{0.38,0.63,0.69}{\textbf{\textit{{#1}}}}}
    \newcommand{\VariableTok}[1]{\textcolor[rgb]{0.10,0.09,0.49}{{#1}}}
    \newcommand{\ControlFlowTok}[1]{\textcolor[rgb]{0.00,0.44,0.13}{\textbf{{#1}}}}
    \newcommand{\OperatorTok}[1]{\textcolor[rgb]{0.40,0.40,0.40}{{#1}}}
    \newcommand{\BuiltInTok}[1]{{#1}}
    \newcommand{\ExtensionTok}[1]{{#1}}
    \newcommand{\PreprocessorTok}[1]{\textcolor[rgb]{0.74,0.48,0.00}{{#1}}}
    \newcommand{\AttributeTok}[1]{\textcolor[rgb]{0.49,0.56,0.16}{{#1}}}
    \newcommand{\InformationTok}[1]{\textcolor[rgb]{0.38,0.63,0.69}{\textbf{\textit{{#1}}}}}
    \newcommand{\WarningTok}[1]{\textcolor[rgb]{0.38,0.63,0.69}{\textbf{\textit{{#1}}}}}


    % Define a nice break command that doesn't care if a line doesn't already
    % exist.
    \def\br{\hspace*{\fill} \\* }
    % Math Jax compatibility definitions
    \def\gt{>}
    \def\lt{<}
    \let\Oldtex\TeX
    \let\Oldlatex\LaTeX
    \renewcommand{\TeX}{\textrm{\Oldtex}}
    \renewcommand{\LaTeX}{\textrm{\Oldlatex}}
    % Document parameters
    % Document title
    \title{Report}
    
    
    
    
    
    
    
% Pygments definitions
\makeatletter
\def\PY@reset{\let\PY@it=\relax \let\PY@bf=\relax%
    \let\PY@ul=\relax \let\PY@tc=\relax%
    \let\PY@bc=\relax \let\PY@ff=\relax}
\def\PY@tok#1{\csname PY@tok@#1\endcsname}
\def\PY@toks#1+{\ifx\relax#1\empty\else%
    \PY@tok{#1}\expandafter\PY@toks\fi}
\def\PY@do#1{\PY@bc{\PY@tc{\PY@ul{%
    \PY@it{\PY@bf{\PY@ff{#1}}}}}}}
\def\PY#1#2{\PY@reset\PY@toks#1+\relax+\PY@do{#2}}

\@namedef{PY@tok@w}{\def\PY@tc##1{\textcolor[rgb]{0.73,0.73,0.73}{##1}}}
\@namedef{PY@tok@c}{\let\PY@it=\textit\def\PY@tc##1{\textcolor[rgb]{0.24,0.48,0.48}{##1}}}
\@namedef{PY@tok@cp}{\def\PY@tc##1{\textcolor[rgb]{0.61,0.40,0.00}{##1}}}
\@namedef{PY@tok@k}{\let\PY@bf=\textbf\def\PY@tc##1{\textcolor[rgb]{0.00,0.50,0.00}{##1}}}
\@namedef{PY@tok@kp}{\def\PY@tc##1{\textcolor[rgb]{0.00,0.50,0.00}{##1}}}
\@namedef{PY@tok@kt}{\def\PY@tc##1{\textcolor[rgb]{0.69,0.00,0.25}{##1}}}
\@namedef{PY@tok@o}{\def\PY@tc##1{\textcolor[rgb]{0.40,0.40,0.40}{##1}}}
\@namedef{PY@tok@ow}{\let\PY@bf=\textbf\def\PY@tc##1{\textcolor[rgb]{0.67,0.13,1.00}{##1}}}
\@namedef{PY@tok@nb}{\def\PY@tc##1{\textcolor[rgb]{0.00,0.50,0.00}{##1}}}
\@namedef{PY@tok@nf}{\def\PY@tc##1{\textcolor[rgb]{0.00,0.00,1.00}{##1}}}
\@namedef{PY@tok@nc}{\let\PY@bf=\textbf\def\PY@tc##1{\textcolor[rgb]{0.00,0.00,1.00}{##1}}}
\@namedef{PY@tok@nn}{\let\PY@bf=\textbf\def\PY@tc##1{\textcolor[rgb]{0.00,0.00,1.00}{##1}}}
\@namedef{PY@tok@ne}{\let\PY@bf=\textbf\def\PY@tc##1{\textcolor[rgb]{0.80,0.25,0.22}{##1}}}
\@namedef{PY@tok@nv}{\def\PY@tc##1{\textcolor[rgb]{0.10,0.09,0.49}{##1}}}
\@namedef{PY@tok@no}{\def\PY@tc##1{\textcolor[rgb]{0.53,0.00,0.00}{##1}}}
\@namedef{PY@tok@nl}{\def\PY@tc##1{\textcolor[rgb]{0.46,0.46,0.00}{##1}}}
\@namedef{PY@tok@ni}{\let\PY@bf=\textbf\def\PY@tc##1{\textcolor[rgb]{0.44,0.44,0.44}{##1}}}
\@namedef{PY@tok@na}{\def\PY@tc##1{\textcolor[rgb]{0.41,0.47,0.13}{##1}}}
\@namedef{PY@tok@nt}{\let\PY@bf=\textbf\def\PY@tc##1{\textcolor[rgb]{0.00,0.50,0.00}{##1}}}
\@namedef{PY@tok@nd}{\def\PY@tc##1{\textcolor[rgb]{0.67,0.13,1.00}{##1}}}
\@namedef{PY@tok@s}{\def\PY@tc##1{\textcolor[rgb]{0.73,0.13,0.13}{##1}}}
\@namedef{PY@tok@sd}{\let\PY@it=\textit\def\PY@tc##1{\textcolor[rgb]{0.73,0.13,0.13}{##1}}}
\@namedef{PY@tok@si}{\let\PY@bf=\textbf\def\PY@tc##1{\textcolor[rgb]{0.64,0.35,0.47}{##1}}}
\@namedef{PY@tok@se}{\let\PY@bf=\textbf\def\PY@tc##1{\textcolor[rgb]{0.67,0.36,0.12}{##1}}}
\@namedef{PY@tok@sr}{\def\PY@tc##1{\textcolor[rgb]{0.64,0.35,0.47}{##1}}}
\@namedef{PY@tok@ss}{\def\PY@tc##1{\textcolor[rgb]{0.10,0.09,0.49}{##1}}}
\@namedef{PY@tok@sx}{\def\PY@tc##1{\textcolor[rgb]{0.00,0.50,0.00}{##1}}}
\@namedef{PY@tok@m}{\def\PY@tc##1{\textcolor[rgb]{0.40,0.40,0.40}{##1}}}
\@namedef{PY@tok@gh}{\let\PY@bf=\textbf\def\PY@tc##1{\textcolor[rgb]{0.00,0.00,0.50}{##1}}}
\@namedef{PY@tok@gu}{\let\PY@bf=\textbf\def\PY@tc##1{\textcolor[rgb]{0.50,0.00,0.50}{##1}}}
\@namedef{PY@tok@gd}{\def\PY@tc##1{\textcolor[rgb]{0.63,0.00,0.00}{##1}}}
\@namedef{PY@tok@gi}{\def\PY@tc##1{\textcolor[rgb]{0.00,0.52,0.00}{##1}}}
\@namedef{PY@tok@gr}{\def\PY@tc##1{\textcolor[rgb]{0.89,0.00,0.00}{##1}}}
\@namedef{PY@tok@ge}{\let\PY@it=\textit}
\@namedef{PY@tok@gs}{\let\PY@bf=\textbf}
\@namedef{PY@tok@gp}{\let\PY@bf=\textbf\def\PY@tc##1{\textcolor[rgb]{0.00,0.00,0.50}{##1}}}
\@namedef{PY@tok@go}{\def\PY@tc##1{\textcolor[rgb]{0.44,0.44,0.44}{##1}}}
\@namedef{PY@tok@gt}{\def\PY@tc##1{\textcolor[rgb]{0.00,0.27,0.87}{##1}}}
\@namedef{PY@tok@err}{\def\PY@bc##1{{\setlength{\fboxsep}{\string -\fboxrule}\fcolorbox[rgb]{1.00,0.00,0.00}{1,1,1}{\strut ##1}}}}
\@namedef{PY@tok@kc}{\let\PY@bf=\textbf\def\PY@tc##1{\textcolor[rgb]{0.00,0.50,0.00}{##1}}}
\@namedef{PY@tok@kd}{\let\PY@bf=\textbf\def\PY@tc##1{\textcolor[rgb]{0.00,0.50,0.00}{##1}}}
\@namedef{PY@tok@kn}{\let\PY@bf=\textbf\def\PY@tc##1{\textcolor[rgb]{0.00,0.50,0.00}{##1}}}
\@namedef{PY@tok@kr}{\let\PY@bf=\textbf\def\PY@tc##1{\textcolor[rgb]{0.00,0.50,0.00}{##1}}}
\@namedef{PY@tok@bp}{\def\PY@tc##1{\textcolor[rgb]{0.00,0.50,0.00}{##1}}}
\@namedef{PY@tok@fm}{\def\PY@tc##1{\textcolor[rgb]{0.00,0.00,1.00}{##1}}}
\@namedef{PY@tok@vc}{\def\PY@tc##1{\textcolor[rgb]{0.10,0.09,0.49}{##1}}}
\@namedef{PY@tok@vg}{\def\PY@tc##1{\textcolor[rgb]{0.10,0.09,0.49}{##1}}}
\@namedef{PY@tok@vi}{\def\PY@tc##1{\textcolor[rgb]{0.10,0.09,0.49}{##1}}}
\@namedef{PY@tok@vm}{\def\PY@tc##1{\textcolor[rgb]{0.10,0.09,0.49}{##1}}}
\@namedef{PY@tok@sa}{\def\PY@tc##1{\textcolor[rgb]{0.73,0.13,0.13}{##1}}}
\@namedef{PY@tok@sb}{\def\PY@tc##1{\textcolor[rgb]{0.73,0.13,0.13}{##1}}}
\@namedef{PY@tok@sc}{\def\PY@tc##1{\textcolor[rgb]{0.73,0.13,0.13}{##1}}}
\@namedef{PY@tok@dl}{\def\PY@tc##1{\textcolor[rgb]{0.73,0.13,0.13}{##1}}}
\@namedef{PY@tok@s2}{\def\PY@tc##1{\textcolor[rgb]{0.73,0.13,0.13}{##1}}}
\@namedef{PY@tok@sh}{\def\PY@tc##1{\textcolor[rgb]{0.73,0.13,0.13}{##1}}}
\@namedef{PY@tok@s1}{\def\PY@tc##1{\textcolor[rgb]{0.73,0.13,0.13}{##1}}}
\@namedef{PY@tok@mb}{\def\PY@tc##1{\textcolor[rgb]{0.40,0.40,0.40}{##1}}}
\@namedef{PY@tok@mf}{\def\PY@tc##1{\textcolor[rgb]{0.40,0.40,0.40}{##1}}}
\@namedef{PY@tok@mh}{\def\PY@tc##1{\textcolor[rgb]{0.40,0.40,0.40}{##1}}}
\@namedef{PY@tok@mi}{\def\PY@tc##1{\textcolor[rgb]{0.40,0.40,0.40}{##1}}}
\@namedef{PY@tok@il}{\def\PY@tc##1{\textcolor[rgb]{0.40,0.40,0.40}{##1}}}
\@namedef{PY@tok@mo}{\def\PY@tc##1{\textcolor[rgb]{0.40,0.40,0.40}{##1}}}
\@namedef{PY@tok@ch}{\let\PY@it=\textit\def\PY@tc##1{\textcolor[rgb]{0.24,0.48,0.48}{##1}}}
\@namedef{PY@tok@cm}{\let\PY@it=\textit\def\PY@tc##1{\textcolor[rgb]{0.24,0.48,0.48}{##1}}}
\@namedef{PY@tok@cpf}{\let\PY@it=\textit\def\PY@tc##1{\textcolor[rgb]{0.24,0.48,0.48}{##1}}}
\@namedef{PY@tok@c1}{\let\PY@it=\textit\def\PY@tc##1{\textcolor[rgb]{0.24,0.48,0.48}{##1}}}
\@namedef{PY@tok@cs}{\let\PY@it=\textit\def\PY@tc##1{\textcolor[rgb]{0.24,0.48,0.48}{##1}}}

\def\PYZbs{\char`\\}
\def\PYZus{\char`\_}
\def\PYZob{\char`\{}
\def\PYZcb{\char`\}}
\def\PYZca{\char`\^}
\def\PYZam{\char`\&}
\def\PYZlt{\char`\<}
\def\PYZgt{\char`\>}
\def\PYZsh{\char`\#}
\def\PYZpc{\char`\%}
\def\PYZdl{\char`\$}
\def\PYZhy{\char`\-}
\def\PYZsq{\char`\'}
\def\PYZdq{\char`\"}
\def\PYZti{\char`\~}
% for compatibility with earlier versions
\def\PYZat{@}
\def\PYZlb{[}
\def\PYZrb{]}
\makeatother


    % For linebreaks inside Verbatim environment from package fancyvrb.
    \makeatletter
        \newbox\Wrappedcontinuationbox
        \newbox\Wrappedvisiblespacebox
        \newcommand*\Wrappedvisiblespace {\textcolor{red}{\textvisiblespace}}
        \newcommand*\Wrappedcontinuationsymbol {\textcolor{red}{\llap{\tiny$\m@th\hookrightarrow$}}}
        \newcommand*\Wrappedcontinuationindent {3ex }
        \newcommand*\Wrappedafterbreak {\kern\Wrappedcontinuationindent\copy\Wrappedcontinuationbox}
        % Take advantage of the already applied Pygments mark-up to insert
        % potential linebreaks for TeX processing.
        %        {, <, #, %, $, ' and ": go to next line.
        %        _, }, ^, &, >, - and ~: stay at end of broken line.
        % Use of \textquotesingle for straight quote.
        \newcommand*\Wrappedbreaksatspecials {%
            \def\PYGZus{\discretionary{\char`\_}{\Wrappedafterbreak}{\char`\_}}%
            \def\PYGZob{\discretionary{}{\Wrappedafterbreak\char`\{}{\char`\{}}%
            \def\PYGZcb{\discretionary{\char`\}}{\Wrappedafterbreak}{\char`\}}}%
            \def\PYGZca{\discretionary{\char`\^}{\Wrappedafterbreak}{\char`\^}}%
            \def\PYGZam{\discretionary{\char`\&}{\Wrappedafterbreak}{\char`\&}}%
            \def\PYGZlt{\discretionary{}{\Wrappedafterbreak\char`\<}{\char`\<}}%
            \def\PYGZgt{\discretionary{\char`\>}{\Wrappedafterbreak}{\char`\>}}%
            \def\PYGZsh{\discretionary{}{\Wrappedafterbreak\char`\#}{\char`\#}}%
            \def\PYGZpc{\discretionary{}{\Wrappedafterbreak\char`\%}{\char`\%}}%
            \def\PYGZdl{\discretionary{}{\Wrappedafterbreak\char`\$}{\char`\$}}%
            \def\PYGZhy{\discretionary{\char`\-}{\Wrappedafterbreak}{\char`\-}}%
            \def\PYGZsq{\discretionary{}{\Wrappedafterbreak\textquotesingle}{\textquotesingle}}%
            \def\PYGZdq{\discretionary{}{\Wrappedafterbreak\char`\"}{\char`\"}}%
            \def\PYGZti{\discretionary{\char`\~}{\Wrappedafterbreak}{\char`\~}}%
        }
        % Some characters . , ; ? ! / are not pygmentized.
        % This macro makes them "active" and they will insert potential linebreaks
        \newcommand*\Wrappedbreaksatpunct {%
            \lccode`\~`\.\lowercase{\def~}{\discretionary{\hbox{\char`\.}}{\Wrappedafterbreak}{\hbox{\char`\.}}}%
            \lccode`\~`\,\lowercase{\def~}{\discretionary{\hbox{\char`\,}}{\Wrappedafterbreak}{\hbox{\char`\,}}}%
            \lccode`\~`\;\lowercase{\def~}{\discretionary{\hbox{\char`\;}}{\Wrappedafterbreak}{\hbox{\char`\;}}}%
            \lccode`\~`\:\lowercase{\def~}{\discretionary{\hbox{\char`\:}}{\Wrappedafterbreak}{\hbox{\char`\:}}}%
            \lccode`\~`\?\lowercase{\def~}{\discretionary{\hbox{\char`\?}}{\Wrappedafterbreak}{\hbox{\char`\?}}}%
            \lccode`\~`\!\lowercase{\def~}{\discretionary{\hbox{\char`\!}}{\Wrappedafterbreak}{\hbox{\char`\!}}}%
            \lccode`\~`\/\lowercase{\def~}{\discretionary{\hbox{\char`\/}}{\Wrappedafterbreak}{\hbox{\char`\/}}}%
            \catcode`\.\active
            \catcode`\,\active
            \catcode`\;\active
            \catcode`\:\active
            \catcode`\?\active
            \catcode`\!\active
            \catcode`\/\active
            \lccode`\~`\~
        }
    \makeatother

    \let\OriginalVerbatim=\Verbatim
    \makeatletter
    \renewcommand{\Verbatim}[1][1]{%
        %\parskip\z@skip
        \sbox\Wrappedcontinuationbox {\Wrappedcontinuationsymbol}%
        \sbox\Wrappedvisiblespacebox {\FV@SetupFont\Wrappedvisiblespace}%
        \def\FancyVerbFormatLine ##1{\hsize\linewidth
            \vtop{\raggedright\hyphenpenalty\z@\exhyphenpenalty\z@
                \doublehyphendemerits\z@\finalhyphendemerits\z@
                \strut ##1\strut}%
        }%
        % If the linebreak is at a space, the latter will be displayed as visible
        % space at end of first line, and a continuation symbol starts next line.
        % Stretch/shrink are however usually zero for typewriter font.
        \def\FV@Space {%
            \nobreak\hskip\z@ plus\fontdimen3\font minus\fontdimen4\font
            \discretionary{\copy\Wrappedvisiblespacebox}{\Wrappedafterbreak}
            {\kern\fontdimen2\font}%
        }%

        % Allow breaks at special characters using \PYG... macros.
        \Wrappedbreaksatspecials
        % Breaks at punctuation characters . , ; ? ! and / need catcode=\active
        \OriginalVerbatim[#1,codes*=\Wrappedbreaksatpunct]%
    }
    \makeatother

    % Exact colors from NB
    \definecolor{incolor}{HTML}{303F9F}
    \definecolor{outcolor}{HTML}{D84315}
    \definecolor{cellborder}{HTML}{CFCFCF}
    \definecolor{cellbackground}{HTML}{F7F7F7}

    % prompt
    \makeatletter
    \newcommand{\boxspacing}{\kern\kvtcb@left@rule\kern\kvtcb@boxsep}
    \makeatother
    \newcommand{\prompt}[4]{
        {\ttfamily\llap{{\color{#2}[#3]:\hspace{3pt}#4}}\vspace{-\baselineskip}}
    }
    

    
    % Prevent overflowing lines due to hard-to-break entities
    \sloppy
    % Setup hyperref package
    \hypersetup{
      breaklinks=true,  % so long urls are correctly broken across lines
      colorlinks=true,
      urlcolor=urlcolor,
      linkcolor=linkcolor,
      citecolor=citecolor,
      }
    % Slightly bigger margins than the latex defaults
    
    \geometry{verbose,tmargin=1in,bmargin=1in,lmargin=1in,rmargin=1in}
    
    

\begin{document}
    
    \maketitle
    
    

    
    \begin{tcolorbox}[breakable, size=fbox, boxrule=1pt, pad at break*=1mm,colback=cellbackground, colframe=cellborder]
\prompt{In}{incolor}{1}{\boxspacing}
\begin{Verbatim}[commandchars=\\\{\}]
\PY{k+kn}{import} \PY{n+nn}{plotly}\PY{n+nn}{.}\PY{n+nn}{express} \PY{k}{as} \PY{n+nn}{px}
\PY{k+kn}{import} \PY{n+nn}{matplotlib}\PY{n+nn}{.}\PY{n+nn}{pyplot} \PY{k}{as} \PY{n+nn}{plt}
\PY{k+kn}{import} \PY{n+nn}{numpy} \PY{k}{as} \PY{n+nn}{np}
\PY{k+kn}{from} \PY{n+nn}{sklearn}\PY{n+nn}{.}\PY{n+nn}{preprocessing} \PY{k+kn}{import} \PY{n}{StandardScaler}
\PY{k+kn}{from} \PY{n+nn}{sklearn}\PY{n+nn}{.}\PY{n+nn}{model\PYZus{}selection} \PY{k+kn}{import} \PY{n}{train\PYZus{}test\PYZus{}split}
\PY{k+kn}{import} \PY{n+nn}{pandas} \PY{k}{as} \PY{n+nn}{pd}
\PY{k+kn}{from} \PY{n+nn}{sklearn}\PY{n+nn}{.}\PY{n+nn}{metrics} \PY{k+kn}{import} \PY{n}{accuracy\PYZus{}score}\PY{p}{,} \PY{n}{classification\PYZus{}report}\PY{p}{,} \PY{n}{confusion\PYZus{}matrix}
\PY{k+kn}{import} \PY{n+nn}{seaborn} \PY{k}{as} \PY{n+nn}{sns}
\end{Verbatim}
\end{tcolorbox}

    \section{Classifying Galaxies in Galaxy
Zoo}\label{classifying-galaxies-in-galaxy-zoo}

    The aim of this project is to train a model to classying galaxies based
on their distinct shapes, according to the \textbf{Galaxy Zoo
Challenge}* on Kaggle. We work on the dataset which consists of 61,578
images with corresponding labels, that represent 37 questions that were
been asked to users about the galaxies images they were looking at.

    Questions Tree

Retrieved from \url{https://arxiv.org/abs/1308.3496}

    \subsection{Exploring Data Analysis}\label{exploring-data-analysis}

In the graph below we show the confidence level of each answerd reported
in the dataset. Notice that each colum of the dataset is named like
``ClassA.B'' where A is the question and B are the different possible
answers.

    Notice that two classes with most responses of high confidence level are
1 and 6. We decided to reduce our features based on this observation and
selected classes 1.1,1.2 and 6.1.

\begin{itemize}
\tightlist
\item
  \textbf{Class1.1}: This galaxy simply smooth and rounded.
\item
  \textbf{Class1.2}: This galaxy has a sign of a disk.
\item
  \textbf{Class6.1} This galaxy is odd.
\end{itemize}

Therefore, we choose these three features to classyfing the shapes of
galaxies, reducing this category into \textbf{elliptical},
\textbf{spiral} or \textbf{odd}.

    First, we have removed the objects that more than 50\% of the
respondents agreed that doesn't represent a galaxy at all. Then we
select for spirals and ellipticals the objects that fall into these two
categories for more than 70\% of the people and also that were not
considered odd by more than 90\% of them. Finally we identified as odd
galaxies the ojects that ere considered unusual by more than 63\%.

We were able to further reduce our data set and by taking 5000 values of
each of the three selected classes so as to be computationally efficient
and start with an homogenous sample, ending up with 15000 rows of sample
data.

    \begin{tcolorbox}[breakable, size=fbox, boxrule=1pt, pad at break*=1mm,colback=cellbackground, colframe=cellborder]
\prompt{In}{incolor}{2}{\boxspacing}
\begin{Verbatim}[commandchars=\\\{\}]
\PY{n}{labels} \PY{o}{=} \PY{n}{pd}\PY{o}{.}\PY{n}{read\PYZus{}csv}\PY{p}{(}\PY{l+s+s2}{\PYZdq{}}\PY{l+s+s2}{Galaxy\PYZus{}data/training\PYZus{}solutions\PYZus{}rev1.csv}\PY{l+s+s2}{\PYZdq{}}\PY{p}{)}
\PY{n}{labels}\PY{o}{=}\PY{n}{labels}\PY{o}{.}\PY{n}{drop}\PY{p}{(}\PY{n}{labels}\PY{p}{[}\PY{n}{labels}\PY{p}{[}\PY{l+s+s1}{\PYZsq{}}\PY{l+s+s1}{Class1.3}\PY{l+s+s1}{\PYZsq{}}\PY{p}{]} \PY{o}{\PYZgt{}} \PY{l+m+mf}{0.5}\PY{p}{]}\PY{o}{.}\PY{n}{index}\PY{p}{)}
\PY{n}{labels} \PY{o}{=} \PY{n}{labels}\PY{p}{[}\PY{p}{[}\PY{l+s+s1}{\PYZsq{}}\PY{l+s+s1}{GalaxyID}\PY{l+s+s1}{\PYZsq{}}\PY{p}{,} \PY{l+s+s1}{\PYZsq{}}\PY{l+s+s1}{Class1.1}\PY{l+s+s1}{\PYZsq{}}\PY{p}{,} \PY{l+s+s1}{\PYZsq{}}\PY{l+s+s1}{Class1.2}\PY{l+s+s1}{\PYZsq{}}\PY{p}{,} \PY{l+s+s1}{\PYZsq{}}\PY{l+s+s1}{Class6.1}\PY{l+s+s1}{\PYZsq{}}\PY{p}{]}\PY{p}{]}
\PY{n}{labels}\PY{p}{[}\PY{l+s+s1}{\PYZsq{}}\PY{l+s+s1}{Result}\PY{l+s+s1}{\PYZsq{}}\PY{p}{]} \PY{o}{=} \PY{l+s+s1}{\PYZsq{}}\PY{l+s+s1}{i}\PY{l+s+s1}{\PYZsq{}}
\PY{n}{labels}\PY{o}{.}\PY{n}{loc}\PY{p}{[}\PY{p}{(}\PY{n}{labels}\PY{p}{[}\PY{l+s+s1}{\PYZsq{}}\PY{l+s+s1}{Class1.1}\PY{l+s+s1}{\PYZsq{}}\PY{p}{]} \PY{o}{\PYZgt{}} \PY{l+m+mf}{0.7} \PY{p}{)} \PY{o}{\PYZam{}} \PY{p}{(}\PY{n}{labels}\PY{p}{[}\PY{l+s+s1}{\PYZsq{}}\PY{l+s+s1}{Class1.2}\PY{l+s+s1}{\PYZsq{}}\PY{p}{]} \PY{o}{\PYZlt{}} \PY{l+m+mf}{0.3} \PY{p}{)} \PY{o}{\PYZam{}} \PY{p}{(}\PY{n}{labels}\PY{p}{[}\PY{l+s+s1}{\PYZsq{}}\PY{l+s+s1}{Class6.1}\PY{l+s+s1}{\PYZsq{}}\PY{p}{]} \PY{o}{\PYZlt{}} \PY{l+m+mf}{0.1} \PY{p}{)}\PY{p}{,} \PY{l+s+s1}{\PYZsq{}}\PY{l+s+s1}{Result}\PY{l+s+s1}{\PYZsq{}}\PY{p}{]} \PY{o}{=} \PY{l+s+s1}{\PYZsq{}}\PY{l+s+s1}{e}\PY{l+s+s1}{\PYZsq{}}
\PY{n}{labels}\PY{o}{.}\PY{n}{loc}\PY{p}{[}\PY{p}{(}\PY{n}{labels}\PY{p}{[}\PY{l+s+s1}{\PYZsq{}}\PY{l+s+s1}{Class1.1}\PY{l+s+s1}{\PYZsq{}}\PY{p}{]} \PY{o}{\PYZlt{}} \PY{l+m+mf}{0.3} \PY{p}{)} \PY{o}{\PYZam{}} \PY{p}{(}\PY{n}{labels}\PY{p}{[}\PY{l+s+s1}{\PYZsq{}}\PY{l+s+s1}{Class1.2}\PY{l+s+s1}{\PYZsq{}}\PY{p}{]} \PY{o}{\PYZgt{}} \PY{l+m+mf}{0.7} \PY{p}{)} \PY{o}{\PYZam{}} \PY{p}{(}\PY{n}{labels}\PY{p}{[}\PY{l+s+s1}{\PYZsq{}}\PY{l+s+s1}{Class6.1}\PY{l+s+s1}{\PYZsq{}}\PY{p}{]} \PY{o}{\PYZlt{}} \PY{l+m+mf}{0.1} \PY{p}{)} \PY{p}{,} \PY{l+s+s1}{\PYZsq{}}\PY{l+s+s1}{Result}\PY{l+s+s1}{\PYZsq{}}\PY{p}{]} \PY{o}{=} \PY{l+s+s1}{\PYZsq{}}\PY{l+s+s1}{s}\PY{l+s+s1}{\PYZsq{}}
\PY{n}{labels}\PY{o}{.}\PY{n}{loc}\PY{p}{[} \PY{p}{(}\PY{n}{labels}\PY{p}{[}\PY{l+s+s1}{\PYZsq{}}\PY{l+s+s1}{Class6.1}\PY{l+s+s1}{\PYZsq{}}\PY{p}{]} \PY{o}{\PYZgt{}} \PY{l+m+mf}{0.63} \PY{p}{)}\PY{p}{,} \PY{l+s+s1}{\PYZsq{}}\PY{l+s+s1}{Result}\PY{l+s+s1}{\PYZsq{}}\PY{p}{]} \PY{o}{=} \PY{l+s+s1}{\PYZsq{}}\PY{l+s+s1}{o}\PY{l+s+s1}{\PYZsq{}}

\PY{n}{elip\PYZus{}df} \PY{o}{=} \PY{n}{labels}\PY{p}{[}\PY{n}{labels}\PY{p}{[}\PY{l+s+s1}{\PYZsq{}}\PY{l+s+s1}{Result}\PY{l+s+s1}{\PYZsq{}}\PY{p}{]} \PY{o}{==} \PY{l+s+s1}{\PYZsq{}}\PY{l+s+s1}{e}\PY{l+s+s1}{\PYZsq{}}\PY{p}{]}
\PY{n}{spiral\PYZus{}df} \PY{o}{=} \PY{n}{labels}\PY{p}{[}\PY{n}{labels}\PY{p}{[}\PY{l+s+s1}{\PYZsq{}}\PY{l+s+s1}{Result}\PY{l+s+s1}{\PYZsq{}}\PY{p}{]} \PY{o}{==} \PY{l+s+s1}{\PYZsq{}}\PY{l+s+s1}{s}\PY{l+s+s1}{\PYZsq{}}\PY{p}{]}
\PY{n}{odd\PYZus{}df} \PY{o}{=} \PY{n}{labels}\PY{p}{[}\PY{n}{labels}\PY{p}{[}\PY{l+s+s1}{\PYZsq{}}\PY{l+s+s1}{Result}\PY{l+s+s1}{\PYZsq{}}\PY{p}{]} \PY{o}{==} \PY{l+s+s1}{\PYZsq{}}\PY{l+s+s1}{o}\PY{l+s+s1}{\PYZsq{}}\PY{p}{]}

\PY{c+c1}{\PYZsh{} Sample 5000 values from each category}

\PY{n}{category1\PYZus{}sampled} \PY{o}{=} \PY{n}{elip\PYZus{}df}\PY{o}{.}\PY{n}{sample}\PY{p}{(}\PY{n}{n}\PY{o}{=}\PY{l+m+mi}{5000}\PY{p}{,} \PY{n}{random\PYZus{}state}\PY{o}{=}\PY{l+m+mi}{42}\PY{p}{)}
\PY{n}{category2\PYZus{}sampled} \PY{o}{=} \PY{n}{spiral\PYZus{}df}\PY{o}{.}\PY{n}{sample}\PY{p}{(}\PY{n}{n}\PY{o}{=}\PY{l+m+mi}{5000}\PY{p}{,} \PY{n}{random\PYZus{}state}\PY{o}{=}\PY{l+m+mi}{42}\PY{p}{)}
\PY{n}{category3\PYZus{}sampled} \PY{o}{=} \PY{n}{odd\PYZus{}df}\PY{o}{.}\PY{n}{sample}\PY{p}{(}\PY{n}{n}\PY{o}{=}\PY{l+m+mi}{5000}\PY{p}{,} \PY{n}{random\PYZus{}state}\PY{o}{=}\PY{l+m+mi}{42}\PY{p}{)}

\PY{c+c1}{\PYZsh{} Concatenate the sampled DataFrames back together}

\PY{n}{sampled\PYZus{}df} \PY{o}{=} \PY{n}{pd}\PY{o}{.}\PY{n}{concat}\PY{p}{(}\PY{p}{[}\PY{n}{category1\PYZus{}sampled}\PY{p}{,} \PY{n}{category2\PYZus{}sampled}\PY{p}{,} \PY{n}{category3\PYZus{}sampled}\PY{p}{]}\PY{p}{)}

\PY{n}{galaxy} \PY{o}{=}  \PY{n}{sampled\PYZus{}df}\PY{o}{.}\PY{n}{sort\PYZus{}values}\PY{p}{(}\PY{n}{by}\PY{o}{=}\PY{l+s+s1}{\PYZsq{}}\PY{l+s+s1}{GalaxyID}\PY{l+s+s1}{\PYZsq{}}\PY{p}{)}
\PY{n}{galaxy\PYZus{}ids} \PY{o}{=} \PY{n}{galaxy}\PY{p}{[}\PY{l+s+s1}{\PYZsq{}}\PY{l+s+s1}{GalaxyID}\PY{l+s+s1}{\PYZsq{}}\PY{p}{]}\PY{o}{.}\PY{n}{to\PYZus{}numpy}\PY{p}{(}\PY{p}{)}
\PY{n}{galaxy}\PY{o}{.}\PY{n}{head}\PY{p}{(}\PY{p}{)}
\end{Verbatim}
\end{tcolorbox}

            \begin{tcolorbox}[breakable, size=fbox, boxrule=.5pt, pad at break*=1mm, opacityfill=0]
\prompt{Out}{outcolor}{2}{\boxspacing}
\begin{Verbatim}[commandchars=\\\{\}]
    GalaxyID  Class1.1  Class1.2  Class6.1 Result
2     100053  0.765717  0.177352  0.000000      e
6     100123  0.462492  0.456033  0.687647      o
16    100263  0.179654  0.818530  0.913055      o
19    100322  0.091987  0.908013  0.000000      s
30    100458  0.820908  0.081499  0.921161      o
\end{Verbatim}
\end{tcolorbox}
        
    \subsubsection{Correlation matrix}\label{correlation-matrix}

The correlation matrix suggests strong negative correlations between
Ellipticals and Spiral (this was expected since one excludes the other
because of the way the dataset was constructed), while the relationships
with Odd galaxies are weaker and in different directions (negative with
Ellipticals and positive with Spirals)

\textless h1 Mean Image

    \subsection{Image Preprocessing}\label{image-preprocessing}

In our resultant dataset, we have notice that the galaxy is almost
perfectly centered in every images and a good percentage of the pixels
around the borders is almost black and does not provide any substantial
informations. Therefore, we decided to cut our sample following this
strategy:

\begin{itemize}
\item
  We decide to work with grayscaled images to reduce the memory
  requirement of storing the all dataset. Each image is stored as an
  array where each element represent the lumonisity intensity of a
  single pixel, that ranges between 0(black) and 255(white).
\item
  We calculate a \textbf{mean image} by averaging the corresponding
  pixels across all samples.
\end{itemize}

\textless h1 Mean Image

\begin{itemize}
\tightlist
\item
  We compute and plot the correlation matrix on the mean image to decide
  which part of the image should be cropped.
\end{itemize}

Correlation Matrix of the Mean Image

From the resultant matrix, we decided to extract the central 256x256
pixels for each images.

Original images

Cropped images

    \subsection{Principal Component
Analysis}\label{principal-component-analysis}

We perform the PCA using scikit-learn module in order to reduce the
number of features of our dataset. We use the \emph{n\_components}
parameter equal to 0.8 to retain 80\% of the variance that needs to be
explained. After this operation, we reduced the number of feauters to
\textbf{20}. This percentage of the variance is the one that allowed us
to reach the best value of accuracy without increasing massively the
number of feauters ( with 70\% we have 9 feauters and 5\% less of
accuracy, with 90\% we have 78 feauters with only 0.5\% more of
accuracy).\\
In the graph below we report the eigenvalues of the PCA respect to the
number of components.

In the plot below we have plot three components of the PCA. Elliptical
and spiral galaxies predominantly occupy distinct regions of the space,
while galaxies classified as ``odd'' exhibit a more dispersed
distribution that meets the other two categories evenly. This behaviour
can be attributed to the distinctive characteristics of ellipticals and
spirals, instead odd galaxies still possess elements shared by both
spirals and ellipticals.

    \begin{tcolorbox}[breakable, size=fbox, boxrule=1pt, pad at break*=1mm,colback=cellbackground, colframe=cellborder]
\prompt{In}{incolor}{3}{\boxspacing}
\begin{Verbatim}[commandchars=\\\{\}]
\PY{k+kn}{from} \PY{n+nn}{IPython}\PY{n+nn}{.}\PY{n+nn}{display} \PY{k+kn}{import} \PY{n}{IFrame}
\PY{n}{IFrame}\PY{p}{(}\PY{n}{src}\PY{o}{=}\PY{l+s+s2}{\PYZdq{}}\PY{l+s+s2}{Images/grafico\PYZus{}3d\PYZus{}interattivo.html}\PY{l+s+s2}{\PYZdq{}}\PY{p}{,} \PY{n}{width}\PY{o}{=}\PY{l+s+s1}{\PYZsq{}}\PY{l+s+s1}{100}\PY{l+s+s1}{\PYZpc{}}\PY{l+s+s1}{\PYZsq{}}\PY{p}{,} \PY{n}{height}\PY{o}{=}\PY{l+m+mi}{800}\PY{p}{)}
\end{Verbatim}
\end{tcolorbox}

            \begin{tcolorbox}[breakable, size=fbox, boxrule=.5pt, pad at break*=1mm, opacityfill=0]
\prompt{Out}{outcolor}{3}{\boxspacing}
\begin{Verbatim}[commandchars=\\\{\}]
<IPython.lib.display.IFrame at 0x7f45902188b0>
\end{Verbatim}
\end{tcolorbox}
        
    \section{Entropy and Symmetry
features}\label{entropy-and-symmetry-features}

    After PCA , we assessed symmetry features across various axes (vertical
and diagonal) for the galaxies so as to reduce the numbers of false
positives and true negatives and to help the model to discern odd
galaxies from ellipticals and spirals.

    \begin{tcolorbox}[breakable, size=fbox, boxrule=1pt, pad at break*=1mm,colback=cellbackground, colframe=cellborder]
\prompt{In}{incolor}{7}{\boxspacing}
\begin{Verbatim}[commandchars=\\\{\}]
\PY{n}{pca} \PY{o}{=} \PY{n}{np}\PY{o}{.}\PY{n}{load}\PY{p}{(}\PY{l+s+s2}{\PYZdq{}}\PY{l+s+s2}{Proccessed\PYZus{}data/pca\PYZus{}fast.npy}\PY{l+s+s2}{\PYZdq{}}\PY{p}{,} \PY{n}{allow\PYZus{}pickle}\PY{o}{=}\PY{k+kc}{True}\PY{p}{)}
\PY{n}{df} \PY{o}{=} \PY{n}{pd}\PY{o}{.}\PY{n}{DataFrame}\PY{p}{(}\PY{n}{pca}\PY{p}{)}
\PY{n}{df}\PY{o}{.}\PY{n}{columns} \PY{o}{=} \PY{n}{df}\PY{o}{.}\PY{n}{iloc}\PY{p}{[}\PY{l+m+mi}{0}\PY{p}{]}
\PY{n}{total\PYZus{}pixels} \PY{o}{=} \PY{n+nb}{len}\PY{p}{(}\PY{n}{df}\PY{o}{.}\PY{n}{columns}\PY{p}{)} 
\PY{n}{header} \PY{o}{=} \PY{p}{[}\PY{l+s+sa}{f}\PY{l+s+s1}{\PYZsq{}}\PY{l+s+s1}{Comp\PYZus{}}\PY{l+s+si}{\PYZob{}}\PY{n}{i}\PY{l+s+si}{\PYZcb{}}\PY{l+s+s1}{\PYZsq{}} \PY{k}{for} \PY{n}{i} \PY{o+ow}{in} \PY{n+nb}{range}\PY{p}{(}\PY{l+m+mi}{1}\PY{p}{,} \PY{n}{total\PYZus{}pixels} \PY{o}{+} \PY{l+m+mi}{1}\PY{p}{)}\PY{p}{]}
\PY{n}{df}\PY{o}{.}\PY{n}{columns} \PY{o}{=} \PY{n}{header}
\PY{n}{new\PYZus{}column} \PY{o}{=}\PY{n}{galaxy}\PY{p}{[}\PY{l+s+s1}{\PYZsq{}}\PY{l+s+s1}{Result}\PY{l+s+s1}{\PYZsq{}}\PY{p}{]}
\PY{n}{new\PYZus{}column} \PY{o}{=} \PY{n}{new\PYZus{}column}\PY{o}{.}\PY{n}{values}\PY{o}{.}\PY{n}{astype}\PY{p}{(}\PY{n+nb}{str}\PY{p}{)}
\PY{n}{df}\PY{o}{.}\PY{n}{insert}\PY{p}{(}\PY{l+m+mi}{0}\PY{p}{,} \PY{l+s+s1}{\PYZsq{}}\PY{l+s+s1}{Shape}\PY{l+s+s1}{\PYZsq{}}\PY{p}{,} \PY{n}{new\PYZus{}column}\PY{p}{)}
\PY{n}{new\PYZus{}column} \PY{o}{=} \PY{n}{np}\PY{o}{.}\PY{n}{load}\PY{p}{(}\PY{l+s+s2}{\PYZdq{}}\PY{l+s+s2}{Proccessed\PYZus{}data/entr.npy}\PY{l+s+s2}{\PYZdq{}}\PY{p}{,}\PY{n}{allow\PYZus{}pickle}\PY{o}{=}\PY{k+kc}{True}\PY{p}{)}
\PY{n}{df}\PY{o}{.}\PY{n}{insert}\PY{p}{(}\PY{l+m+mi}{1}\PY{p}{,} \PY{l+s+s1}{\PYZsq{}}\PY{l+s+s1}{Entropy}\PY{l+s+s1}{\PYZsq{}}\PY{p}{,} \PY{n}{new\PYZus{}column}\PY{p}{)}
\PY{n}{new\PYZus{}column} \PY{o}{=} \PY{n}{np}\PY{o}{.}\PY{n}{load}\PY{p}{(}\PY{l+s+s2}{\PYZdq{}}\PY{l+s+s2}{Proccessed\PYZus{}data/entr.npy}\PY{l+s+s2}{\PYZdq{}}\PY{p}{,}\PY{n}{allow\PYZus{}pickle}\PY{o}{=}\PY{k+kc}{True}\PY{p}{)}
\PY{n}{df}\PY{o}{.}\PY{n}{insert}\PY{p}{(}\PY{l+m+mi}{2}\PY{p}{,} \PY{l+s+s1}{\PYZsq{}}\PY{l+s+s1}{Vert\PYZus{}Symm}\PY{l+s+s1}{\PYZsq{}}\PY{p}{,} \PY{n}{new\PYZus{}column}\PY{p}{)}
\PY{n}{new\PYZus{}column} \PY{o}{=} \PY{n}{np}\PY{o}{.}\PY{n}{load}\PY{p}{(}\PY{l+s+s2}{\PYZdq{}}\PY{l+s+s2}{Proccessed\PYZus{}data/entr.npy}\PY{l+s+s2}{\PYZdq{}}\PY{p}{,}\PY{n}{allow\PYZus{}pickle}\PY{o}{=}\PY{k+kc}{True}\PY{p}{)}
\PY{n}{df}\PY{o}{.}\PY{n}{insert}\PY{p}{(}\PY{l+m+mi}{3}\PY{p}{,} \PY{l+s+s1}{\PYZsq{}}\PY{l+s+s1}{Diag\PYZus{}Symm}\PY{l+s+s1}{\PYZsq{}}\PY{p}{,} \PY{n}{new\PYZus{}column}\PY{p}{)}
\PY{n}{df}\PY{o}{.}\PY{n}{head}\PY{p}{(}\PY{p}{)}
\end{Verbatim}
\end{tcolorbox}

            \begin{tcolorbox}[breakable, size=fbox, boxrule=.5pt, pad at break*=1mm, opacityfill=0]
\prompt{Out}{outcolor}{7}{\boxspacing}
\begin{Verbatim}[commandchars=\\\{\}]
  Shape   Entropy  Vert\_Symm  Diag\_Symm       Comp\_1       Comp\_2  \textbackslash{}
0     e  0.025132   0.025132   0.025132  4346.050167 -1456.188393
1     o  0.023503   0.023503   0.023503   889.894082    90.320612
2     o  0.023948   0.023948   0.023948  1568.656494 -1966.178868
3     s  0.017978   0.017978   0.017978 -3270.449328 -1755.345616
4     o  0.023738   0.023738   0.023738   251.866158   992.014197

        Comp\_3       Comp\_4       Comp\_5       Comp\_6  {\ldots}      Comp\_11  \textbackslash{}
0 -1896.784438  -676.409366    36.773572   195.571665  {\ldots}  -227.055876
1  1142.557293   268.943849  1152.781022  -798.075256  {\ldots}   102.786704
2 -1025.526464   686.554955  -964.185359  3750.701429  {\ldots}  1403.017487
3  2012.692208   245.920395   170.642006    69.427892  {\ldots}  -403.672541
4 -1065.144293  1611.747247 -1334.036183 -2011.712083  {\ldots}   770.157136

      Comp\_12      Comp\_13     Comp\_14     Comp\_15      Comp\_16     Comp\_17  \textbackslash{}
0   25.334820    33.682129   57.169679 -255.691422  -112.776652 -293.350937
1   88.255408    82.456882 -471.795176  148.323664 -1485.647633 -418.415224
2  439.064731  1609.516629 -148.428221  297.017676  -579.905823  576.214745
3  -83.690110     0.764526 -164.029216 -334.458436  -151.886122  300.297738
4  206.304658 -1367.498712 -128.088789  404.980605 -1425.601187  -71.209145

      Comp\_18     Comp\_19     Comp\_20
0  392.706494 -121.912697 -102.918707
1 -468.584332  383.944222    7.113392
2  193.280893   28.143759   -5.141216
3  667.889421 -177.189337 -421.773141
4 -408.029391 -430.674039  277.177246

[5 rows x 24 columns]
\end{Verbatim}
\end{tcolorbox}
        
    \subsection{Build the classifier using Random
Forest}\label{build-the-classifier-using-random-forest}

    We have implement Random Forest Hyperparameter Tuning using Sklearn that
help to fine-tune the models. The \emph{Shape} column will be the target
to be predicted. Employing \textbf{train\_test\_split}, we partition the
dataset into test and train sets, with a ratio of 20\% for testing and
80\% for training. Then, we have used \textbf{GridSearchCV} to identify
the optimal parameters for the model.

These are the hyperparameter we have used in the grid:

\begin{itemize}
\item
  \textbf{n\_estimators}: The number of trees in the forest. It takes
  values from the list (50, 100, 150, 200, 300).
\item
  \textbf{max\_depth}: The maximum depth of the trees. It takes values
  from the list (None, 10, 20). A deeper tree can capture more complex
  relationships but may lead to overfitting.
\item
  \textbf{min\_samples\_split}: The minimum number of samples required
  to split an internal node. It takes values from the list (2, 5, 10).
\item
  \textbf{min\_samples\_leaf}: The minimum number of samples required to
  be at a leaf node. It takes values from the list (1, 2, 4).
\item
  \textbf{max\_features}: The number of features to consider when
  looking for the best split. It takes values from the list (`auto',
  `sqrt', `log2').
\end{itemize}

These are the accuracy report with the best parameters obtained by
GridSearchView:

\begin{itemize}
\tightlist
\item
  max\_depth = \textbf{20}
\item
  max\_features = \textbf{log2}
\item
  min\_samples\_leaf = \textbf{1}
\item
  min\_samples\_split = \textbf{2}
\item
  n\_estimators = \textbf{300}
\end{itemize}

    \section{Accuracy of the Classifier}\label{accuracy-of-the-classifier}

In this section we report the accuracy report of our analysis. The best
accuracy achieved is \textasciitilde{}\textbf{82\%}. As we expected, the
odd shape is the most difficult to identify for our model.

    \begin{tcolorbox}[breakable, size=fbox, boxrule=1pt, pad at break*=1mm,colback=cellbackground, colframe=cellborder]
\prompt{In}{incolor}{8}{\boxspacing}
\begin{Verbatim}[commandchars=\\\{\}]
\PY{k}{with} \PY{n+nb}{open}\PY{p}{(}\PY{l+s+s1}{\PYZsq{}}\PY{l+s+s1}{classification\PYZus{}results.txt}\PY{l+s+s1}{\PYZsq{}}\PY{p}{,} \PY{l+s+s1}{\PYZsq{}}\PY{l+s+s1}{r}\PY{l+s+s1}{\PYZsq{}}\PY{p}{)} \PY{k}{as} \PY{n}{file}\PY{p}{:}
    \PY{n}{saved\PYZus{}results} \PY{o}{=} \PY{n}{file}\PY{o}{.}\PY{n}{read}\PY{p}{(}\PY{p}{)}

\PY{n+nb}{print}\PY{p}{(}\PY{n}{saved\PYZus{}results}\PY{p}{)}
\end{Verbatim}
\end{tcolorbox}

    \begin{Verbatim}[commandchars=\\\{\}]
Accuracy: 0.82

Classification Report:
              precision    recall  f1-score   support

           e       0.83      0.84      0.83      1014
           o       0.79      0.80      0.79      1001
           s       0.82      0.80      0.81       985

    accuracy                           0.81      3000
   macro avg       0.81      0.81      0.81      3000
weighted avg       0.81      0.81      0.81      3000



    \end{Verbatim}

    \subsection{Confusion matrix}\label{confusion-matrix}

From the confusion matrix, we can assest that they highest number of
\textbf{false positive} (123 galaxies) are the spiral galaxies that were
mistakenly classified as odd, while the majority of \textbf{false
negative} (101 galaxies) are the elliptical galaxies that were
mistakenly classified as spiral. Possible explanations to this behaviour
are the following:

\begin{itemize}
\item
  Spiral galaxies often exhibit intricate structures, arms, and
  irregularities that may resemble the features associated with odd
  galaxies. The classifier may struggle to differentiate between certain
  types of spiral and odd galaxies. Additionally, the presence of
  unusual or asymmetric features in some spiral galaxies could
  contribute to misclassifications as odd.
\item
  Elliptical galaxies are characterized by their smooth and featureless
  appearance, lacking the prominent arms seen in spiral galaxies. The
  misclassification of elliptical galaxies as spiral may occur when
  there are subtle details or variations in brightness that resemble
  spiral structures.
\end{itemize}

\textless img tit§le=``Conf\_matrix''
src=``Images/Confus\_matrix.png''\textgreater{}


    % Add a bibliography block to the postdoc
    
    
    
\end{document}
